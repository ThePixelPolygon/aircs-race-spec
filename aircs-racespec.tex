\documentclass{article}
\usepackage[T1]{fontenc}
\usepackage[utf8]{inputenc}
\usepackage{xcolor}
\usepackage{tgheros}
\usepackage{mathtools, listings}
\usepackage{fancyhdr}
\usepackage{hyperref}

\lstset{
basicstyle=\small\ttfamily,
columns=flexible,
breaklines=true
}
\pagestyle{fancy}
\renewcommand*\familydefault{\sfdefault}
\newcommand{\defterm}[2]{\textbf{#1} #2}

\title{The \textbf{{\color{red}Air}CS} Race Specification}
\author{The AirCS Foundation}

\begin{document}
\maketitle
\tableofcontents

\section{Introduction}
The AirCS race has become one of the biggest simultaneous-player events on the bits \& Bytes Minecraft 1.x servers. Starting with the second edition of AirCS Race, the AirCS Foundation has installed new tracking systems in all AirCS and SQTR stations to automate the race (henceforth referred to as RaceCS or the RaceCS system). In this document, we will explain how the system works, and outline everything that you need to know to make sure your new station is ready for the next AirCS race.

\pagebreak

\section{The System}
The RaceCS system includes two components: the in-game score system, and the web interface. These components work together to provide the smoothest possible racing experience. They are designed to keep track of where all players have been, and to inform everybody, all racers and spectators, of each player's progress in the race.

\subsection{In-Game Score System}
The in-game system is designed to ensure that the RaceCS system is not dependent on any additional plugins to run at all. Should an offline copy of the server be hosted via LAN or on a server where the RaceCS plugin is not available, a limited version of the system (RaceCS Lite) may be used instead. Each player's score is displayed on the sidebar, and the system makes use of scoreboard tags to ensure no station is counted twice. It also makes sure that only racers are tracked. In addition to this, notifications are sent to all players whenever any player reaches a unique station.

\subsection{Web Interface}
The web interface provides a more detailed view of the race in progress. In addition to providing score updates for those who are not able to view them in-game, it also keeps a history of all notifications sent from when the page was first opened. Users may also opt in to receive push notifications on their devices so they do not have to keep the page open at all times.

Apart from the notifications, the web interface also keeps track of exactly what stations everyone has visited, and what stations remain. This also serves useful to players, as the system will inform them of what stations they may have missed, so they may create more informed strategies to complete the race. 

\pagebreak

\section{Installation}
We now require that for any station to be included in the AirCS race, regardless of what transit system they operate on, they must comply with the AirCS race specification. The station must account for all possible methods of entry, and should be thorougly tested to ensure correct operation.

\textbf{NOTE: } From this point forward, the acronym \textbf{NA} will refer to the shorthand code of your station.
\subsection{Standard Lines}
A standard line refers to the standard track and cart system, the most common method of physical transportation. In summary, a standard line uses a detector rail placed as close to the end of the line as possible. This rail triggers a series of command blocks which accomplishes the following (assuming that the player has not yet reached the station previously):
\begin{itemize}
    \item Announces to all players that an individual has reached a unique station
    \item Increments the player's score by 1
    \item Adds a tag to indicate the player has visited the station
\end{itemize}
To accomplish this, replace a single rail (ideally as close to the end of the line as possible) with a detector rail. Then, place these commands in a chain of command blocks.
\begin{lstlisting}
    /testfor @a[dy=3,tag=!NA]
    /tellraw @a[team=acsrace] ["",{"text":"Air","bold":true,"color":"dark_red"},{"text":"CS","bold":true},{"text":" Race Update: ","bold":true,"color":"gold"},{"selector":"@p"},{"text":" has reached [NAME] station!"}]
    /scoreboard players tag @a[team=acsrace,dy=5] add NA
    /scoreboard players add @a[team=acsrace,dy=6] acsrace 1
    /racecs arrived @a[team=acsrace,dy=7] NA
\end{lstlisting}
\textbf{NOTE:} We have written these selectors assuming that your command blocks will be placed underneath the tracks. Should there be a situation where placing these command blocks underneath the tracks is not possible, adjust the dx, dy, and dz attributes accordingly.
\subsection{SQTR Bullet Boat System}
If the SQTR system connects to a station via the bullet boat, the current ruleset of the AirCS race stipulate that the player \textbf{must step into the station} to clear the checkpoint. In this case, the same set of commands may be used, however, some adjustments will need to be made to them.
\begin{itemize}
    \item The command blocks must be placed at the very left edge of the station (facing towards the station landing)
    \item A line of pressure plates must be installed across the span of the station, with a line of redstone attached to the command blocks (use repeaters where necessary)
    \item The \textit{first} command block in the chain must be set to \textbf{repeat mode.} This will ensure that if more than one person enters, all players will be accounted for.
    \item \textbf{Most importantly, ensure that the selectors are adjusted such that the detection area always spans the width of the station's landing.}
\end{itemize}

\subsection{When to Place Checkpoints}
To save time, you may consider taking some shortcuts when installing RaceCS checkpoints at your station. It may sometimes be unnecessary to install checkpoints on both ends of the track as there may be only one possible method of entry. Here are some tips to help save time and avoid confusion.

\medskip

\textbf{Orphaned stations only require a single checkpoint.} If your target station is an orphaned station, you may place a single checkpoint at the orphaned station. In the case of a doubly-orphaned station, you also do not require checkpoints at both ends of the line.

\medskip

\textbf{Any lines that lead back to AirCS Central Station only require one checkpoint.} AirCS Central Station (referred to in-game as ``Station Air \textbf{Central Hub}'') does not count for score at any point in the game. Therefore, if the destination has a line leading back to central, only the destination requires a checkpoint.

\medskip

\textbf{More often than not, SQTR stations will require checkpoints at both ends of a line.} This is because due to the way that SQTR operates, one may traverse a single line in two directions. Moreover, a majority of SQTR destinations are also shared with AirCS, which counts as a single point regardless of the system used.
\subsection{A note on orbis. Q and AirCS 255}
For the time being, it is against the rules to use orbis. Q by AirCS and AirCS 255. However, when the AirCS 255 system opens, its speed will be reduced significantly for game balance. More details TBD.

\subsection{Deregistration}
Due to the limitations of the Minecraft scoreboard, any new station created that integrates with the RaceCS system must have their scoreboard tags removed one at a time. Beneath the 0th floor of AirCS central station, there is a long chain of command blocks designed to do this. 
\begin{enumerate}
    \item With a WorldEdit wand equipped, select the last two command blocks and move them one block forward.
    \item In the gap created between the command blocks, place a chained, always active command block. The command should be set to the following:
    \begin{lstlisting}
        /execute @e[type=Armor_Stand,name=acsr] ~ ~ ~ scoreboard players tag @p remove NA
    \end{lstlisting}
\end{enumerate}

\pagebreak

\section{Glossary of Terms}
\begin{itemize}
    \item \defterm{Orphaned Station}{A station that is only accessible by exactly one branch of one metro service (also known as a dead-end station). Some examples include (but are not limited to)}
    \begin{itemize}
        \item Central $\rightarrow$ Whale City Central $\rightarrow$ Whale City Commercial
        \item Central $\rightarrow$ Long Island $\rightarrow$ UWGK
    \end{itemize}
    \item \defterm{Doubly Orphaned Station}{An orphaned station that is connected to an orphaned station (see ``orphaned station'')}
    \item \defterm{SQTR}{An alternate transit system that may optionally be used in an AirCS race}
    \item \defterm{SQTR Bullet Boat}{A contained ice and boat system that requires manual user interaction}
    \item \defterm{orbis. Q}{A hands-free hyperloop system}
    \item \defterm{AirCS 255}{A contained speed effect system that requires manual user interaction}
\end{itemize}

\pagebreak

\section{Additional Information}
If you still have any questions regarding RaceCS system installation, contact \textbf{The Pixel Polygon\#2069} on Discord.

\medskip

If you would like to contribute to this document, you may add your name in the credits below and submit a pull request. The original repo may be found at: \url{https://github.com/ThePixelPolygon/aircs-race-spec}.

\medskip

If you would like to contribute translations, you may visit: \url{https://translate.vicr123.com/}.

\pagebreak

\section{Contributors}
Developers:
\begin{itemize}
    \item JPlexer
    \item The Pixel Polygon
    \item Victor Tran
\end{itemize}
Authors:
\begin{itemize}
    \item The Pixel Polygon
\end{itemize}
\end{document}